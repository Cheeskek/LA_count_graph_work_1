\section{Задание 2. Преобразование координат и базиса.}

\textbf{Условие.}\\
В линейном пространстве со стандартным базисом $E = \Set{e_1, e_2, e_3}$, где
\[e_1 = (1, 0, 0), e_2 = (0, 1, 0), e_3 = (0, 0, 1),\]
заданы системы векторов $A = \Set{a_1, a_2, a_3}$ и $B = \Set{b_1, b_2, b_3}$.\\
$A = \Set{a_1, a_2, a_3}\\
a_1 = (0.3536; 0.9268; 0.1268)\\
a_2 = (-0.6124; 0.1268; 0.7803)\\
a_3 = (0.7071; -0.3536; 0.6124)\\
B = \Set{b_1, b_2, b_3}\\
b_1 = (-0.8712; -1.0267; 2.0462)\\
b_2 = (1.9319; 1.5999; -1.307)\\
b_3 = (-2.3801; 2.1143; -0.93)\\
x_B = \begin{pmatrix}2.6 \\ 1.7 \\ 1.2\end{pmatrix}
$
\begin{enumerate}
    \item Покажите, что каждая система образует базис.
    \item Проверьте каждый из этих базисов на ортогональность и нормированность.
    \item Найдите матрицу перехода $T$ из базиса $A$ в базис $B$.
    \item Вектор $x$ в базисе $B$ имеет координаты $x_B = (x_1^\prime, x_2^\prime, x_3^\prime)^T$.
    Найдите его координаты $x_A$ в базисе $A$.
    \item В базисе $E$ изобразите векторы базиса $A$ и вектор $x$.
\end{enumerate}
\vspace{10mm}
\noindent\textbf{Решение.}\\
It is empty but you can fill it!!!

\clearpage
